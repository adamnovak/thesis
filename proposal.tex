%% uctest.tex 11/3/94
%% Copyright (C) 1988-2004 Daniel Gildea, BBF, Ethan Munson.
%
% This work may be distributed and/or modified under the
% conditions of the LaTeX Project Public License, either version 1.3
% of this license or (at your option) any later version.
% The latest version of this license is in
%   http://www.latex-project.org/lppl.txt
% and version 1.3 or later is part of all distributions of LaTeX
% version 2003/12/01 or later.
%
% This work has the LPPL maintenance status "maintained".
% 
% The Current Maintainer of this work is Daniel Gildea.

\documentclass[11pt,proposal]{ucthesis}
\def\dsp{\def\baselinestretch{2.0}\large\normalsize}
\dsp

% 2010june01 sol katzman:
% package geometry should override the various margin settings from .clo and .cls
% and also eliminates issues where the default papersize is A4
\usepackage[letterpaper, left=1.5in, right=1.25in, top=1.25in, bottom=1.25in, includefoot]{geometry}

\begin{document}

% Declarations for Front Matter

\title{Infrastructure for Scalable Analysis of Genomic Variation}
\author{Adam M. Novak}
\degreeyear{2014}
\degreemonth{June}
\degree{DOCTOR OF PHILOSOPHY}
\chair{Professor David Haussler}
\committeememberone{Professor Ivory Insular}
\committeemembertwo{Professor General Reference}
\committeememberthree{Professor Ipsum Lorem}
\numberofmembers{4} %% (including chair) possible: 3, 4, 5, 6
\deanlineone{Dean John Doe}
\deanlinetwo{Vice Provost and Dean of Graduate Studies}
\deanlinethree{}
\field{Bioinformatics}
\campus{Santa Cruz}

\begin{frontmatter}

\maketitle
\copyrightpage

\tableofcontents
\listoffigures
\listoftables

\begin{abstract}
Theses have elements.  Isn't that nice?

\end{abstract}

\begin{acknowledgements}

\end{acknowledgements}

\end{frontmatter}

\chapter{Introduction}

% Biology only works at scale

    % Scale in population and in time is what allows evolution to work
    
    % Scale produces drug resistance in bacteria
    
    % Scale produces cancer in healthy people
    
    % We won't be able to understand biology until we can match its scale

% High throughput is along the wrong dimension (so far)
    
    % We have a lot of features of a few samples.
    
    % To some extent we can get by with clever algorithms and educated guesses as prior knowledge.

% We need to be able to handle millions or even billions of samples if we want to be able to do biology just by looking.

    % Cancer, for example, won't make sense until we do this
    
% We need to be able to do this with something not millions of times better than current computing hardware

     % Because the transistors are now nearly as small as the stuff we're sequencing

% We can't scale that big by pushing BAM files around.

    % A billion 100-gig BAM files is 93 exabytes. That is too big to download.

% 1 in a million things start to happen, and we need infrastructure to handle them
    
    % We need to handle overlapping variants
    
    % We can't just throw out the alternate haplotypes
    
    % Assuming that everyone is contiguous in reference genome coordinates is going to get us into trouble
    
    % We can't keep messing around with the reference coordinates because we can't keep re-mapping every single read ever read.
    
    % We'll want to look at important variation in repetitive areas, and we may not yet have long reads

% So I'm going to build some stuff we will need in order to make this work
    
    % A non-linear coordinate system with stable base identifiers
    
    % A reference graph
    
        % Can capture more than just a single reference sequence
        
        % Reduce mapping bias
        
        % Probably less racist
        
    % A deterministic way to identify observed bases within that graph
        
    % A hierarchy of different versions of this graph with differing levels of specificity
        
        % Interlinked so you can project up and down for a rigorous notion of multimaping
        
% Then I'm going to try it out and discover something useful.
        
\chapter{Background}

% How bioinformatics in general works

    % Download data
    
    % Analyze data
    
    % Already a huge pain with TCGA
    
        % Anyone who wants to work with the data needs a place to locally mirror it
        
        % And that place has to meet TCGA's exacting standards for data security and access control.
        
        % It would be much easier if TCGA could just do all the access control itself.
        
    % This is not going to be able to scale much bigger. Already it's very difficult for smaller labs to use these huge community data sets.

% How genomics works

     % We have The Human Reference Genome
     
     % This gives us a linear contig-base coordinate space
     
     % We map reads to and define genes on that coordinate space
     
        % And define "variation" as any deviation from that sequence
        
        % We have mapping bias which is probably racist
     
     % Everybody uses it, and when we update it everybody has to upgrade
     
        % All the coordinates change
        
        % Published coordinates from back in the day no longer work and need to be lifted over or even remapped

% HG38 changes

    % Version jump
    
    % Lots of alt haplotypes
    
    % Plausible centromeres
    
% How variant analysis works

    % Make big VCFs
    
    % Store all the records as version of a variant at a coordinate position.
    
    % Ignore the fact that people have alternate haplotypes
    
        % No way to specify or look up which haplotype a person has
        
        % Only some coordinate positions are defined for any given person, so some variants are nonsensical for some people.
        
    % The same variant can be called different ways depending on the aligner
    
    % And the same variant can be written in any of three different syntaxes

        % Each of which results in completely different parser data structures available to analysis code
        
    % VCF creates complexity where it isn't and turns holes in the abstractions into pitfall traps for unwary bioinformaticists.
    
% So the old way of doing things is fraying around the edges

% We think we can make a better way using a few key technologies

    % We want to make a graph reference
        
        % We already have haplotypes bubbling off, so we really have a graph already
        
        % Benedict did all this previous work on graph genomes
        
    % We want to put this in an aligned hierarchy
        
        % Like something out of HAL, but with a different interpretation.
    
    % We want to employ FMD-index technology and succinct data structures to map to it uniquely at any given level
    
    % We want to employ Spark/GraphX to build a system to traverse and query these graphs
    
    % We want to employ Avro to give this system a consistent API accessible from a variety of languages.
    
% When this is done, I anticipate it being useful for things like NOTCH2NL
    
    % NOTCH2NL used to be annotated as one gene in hg19
    
    % In hg38 we've worked out that it is really 4 similar genes, -A through -D.
    
    % Now we have 4 very similar genes, near each other, and we want to look at variation in them.
    
    % We have all this microarray data and sequencing data that's super hard to pin down to one and only one paralog.
    
    % To get something out of this region, we need a way to work with multimapped data in a rigorous way.

\chapter{Preliminary Work}

% I have already some some work on infrastructure for the scalable analysis of genomic variation.

\section{Project \#1: Tumor Map Visualization}

% I made a pan-cancer visualization of the entire cross-tissue TCGA data set
    
    % Lay out all the tumors by a similarity metric, and squish them into a hexagonal grid in 2D
    
    % Let people fly around it with Google Maps
    
    % Then draw map overlays depicting various tumor or patient attributes.

    % Less a scaling up of the analysis than a scaling down of the data set into a space where people are comfortable working.
    
% See the paper. 

\section{Project \#2: NOTCH2NL Copy Number Variation Analysis}

% I made a first pass at analyzing the NOTCH2NL region
    
    % I specifically worked on a linear programming based approach to make copy number calls on our corrected assembly using CGH microarray probes originally designed for an assembly that conflated the NOTCH2NL genes
    
    % Trying to handle multimapping, because if I threw out multimapping I would have thrown out XX% of the data.
        
        % This again isn't really scaling, but it's something we need to do to support scaling.
    
% We found that CNV in the NOTCH2NL region can explain disease.

    % See the paper.

\section{Project \#3: Mapping to a Reference Genome Structure}

% We developed a rigorous theory for making reference graphs, putting them in hierarchies, and mapping to them based on context.

% I'm going to need this in my system as an alternative to linear coordinates, since linear coordinates have all the problems mentioned above.

% The hierarchy stuff is also how I plan to formalize multimapping into something useful.

% See the paper.

\chapter{Proposed Work}

\section{Specific Aim \#1: Develop Reference Hierarchy Theory}

% I need a rigorous theory of reference structures, reference hierarchies, mapping, and multimapping

% Also a mathematics of variants as paths, and a reference as a collection of equal variants.

% This is mostly completed.

\section{Specific Aim \#2: Engineer Scalable Software Tools}

% I need to turn Specific Aim #1 from math into software.

% To be minimally useful it needs to be able to do at least what we do now: map to hg38 plus alt haplotypes
    
    % But in a graph-based way
    
% What I actually want to do is to be able to map to hg38 and alts, plus a bunch of other variants.

    % I'd like to show that this reduces reference bias in mapping
    
% I don't plan to scale to more than tens of genomes' worth of sequence in the reference within the scope of this project.

    % I can fit a bunch of examples of variants in without full genomes to hold them.
    
% I'm going to stick an API on it so the software is accessible from multiple languages

    % Ideally with RPC so you don't need to mirror this whole graph structure in order to play with it.

\section{Specific Aim \#3: Discover Biologically Relevant Variation}

% Once I've built the system, I'll browse around the graph and see if I can find anything interesting.

% Since this system will be well-designed for analyzing structural variants, I will see if I can compile some broad statistics about the number and type of structural variants in my reference versus in other compendiums of structural variation.

% If I can't find anything else, I could take another look at the NOTCH2NL region, and repeat my analysis of the CGH microarray data in graph terms.

\appendix
\chapter{Some Ancillary Stuff}

Ancillary material should be put in appendices, which appear BEFORE the
bibliography. 

% %%%%%%%%%%%%%%%%%%%%%%%%%%%%%%%%%%%%%%%%%%%%%%%%%%%%%%%%%
% bibliography

% 2010june01 sol katzman:
% if \nocite is specified, all entries in the bib file are included,
% probably not what you want, so comment out the \nocite and only get the cited references.
\nocite{*}

% 2010june01 sol katzman:
% this makes the bibliography single spaced, with double spacing between entries
\def\baselinestretch{1.0}\large\normalsize

\bibliographystyle{plain}
\bibliography{proposal}

\end{document}
