% Declarations for Front Matter

\title{Infrastructure for Scalable Analysis of Genomic Variation}
\author{Adam M. Novak}
\degreeyear{2017}
\degreemonth{June}
\degree{DOCTOR OF PHILOSOPHY}
\chair{Professor Josh Stuart}
\committeememberone{Distinguished Professor David Haussler}
\committeemembertwo{Assistant Professor Ed Green}
\committeememberthree{Associate Professor Beth Shapiro}
\numberofmembers{4} %% (including chair) possible: 3, 4, 5, 6
\deanlineone{Dean \textbf{Tyrus Miller}}
\deanlinetwo{Vice Provost and Dean of Graduate Studies}
\deanlinethree{}
\field{Bioinformatics}
\campus{Santa Cruz}

\begin{frontmatter}

\maketitle
\copyrightpage

\tableofcontents
\listoffigures
\listoftables

\begin{abstract}
    Bioinformatics thinks too small. If we are to effectively engage with human biology, we need to be able to think on scales of millions or billions of individuals. Unfortunately, the software tools that we use to analyze the human genome, and indeed the very concept of a unified ``the human genome'', start to break down at such scales. In order to remedy this, I propose a three-part research agenda: creating a new conceptual framework for using graphs as pluralistic genomic references, implementing scalable software tools for shared infrastructure to build and operate on such graphs, and applying these tools to solve previously difficult problems in genomics.
\end{abstract}

\begin{acknowledgements}
In addition to my advisor Prof. David Haussler, and the other members of my committee, I would like to thank Dr. Benedict Paten for helping to direct my research efforts, Yohei Rosen for collaborating with me on mapping algorithms, Frank Nothaft for his help with Spark and Avro, and everyone on the NOTCH2NL project. I would also like to thank Anna Henderson for her contributions as an editor.
\end{acknowledgements}

\end{frontmatter}
